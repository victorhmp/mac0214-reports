\documentclass[10pt,twoside,a4paper]{article}
% ^- openany - open new pages in odd/even page

%packages
\usepackage[T1]{fontenc}
\usepackage[utf8]{inputenc}      % Encoding
\usepackage[portuguese]{babel}   % Correção
%\usepackage{caption}             % Legendas
\usepackage{enumerate}

% Matemática
\usepackage{amsmath}             % Matemática
\usepackage{amsthm, amssymb}     % Matemática
\newtheorem*{def*}{Definição}
\newtheorem*{invariant}{Invariante}

% Gráficos
\usepackage[usenames,dvipsnames]{color}  % Cores
\usepackage[pdftex]{graphicx}   % usamos arquivos pdf/png como figura
\usepackage[usenames,svgnames,dvipsnames]{xcolor}

% Desenhos
\usepackage{tikz}
\usepgfmodule{decorations}
\usetikzlibrary{patterns}
\usetikzlibrary{decorations.shapes}
\usetikzlibrary{shapes.geometric}
\usetikzlibrary{decorations.text}
\usetikzlibrary{positioning} % Adjust grid size

% Código-fonte
\usepackage[noend]{algpseudocode}
\usepackage{algorithm}

% Configurações da página
\usepackage{fancyhdr}           % header & footer
\usepackage{float}
\usepackage{setspace}           % espaçamento flexível
\usepackage{indentfirst}        % Identa primeiro parágrafo
\usepackage{makeidx}
\usepackage[nottoc]{tocbibind}  % acrescentamos a  bibliografia/indice/
                                % conteudo no Table of Contents
                                
% Fontes
%\usepackage{helvet}
\renewcommand{\familydefault}{\sfdefault}
\usepackage{type1cm}            % fontes realmente escaláveis
\usepackage{titletoc}
\usepackage{pdflscape}          % Páginas em paisagem
\usepackage{pdfpages}

% Fontes e margens
\usepackage[fixlanguage]{babelbib}
\usepackage[font=small,format=plain,labelfont=bf,up,textfont=it,up]{caption}
\usepackage[a4paper,top=3.0cm,bottom=3.0cm,left=2.0cm,right=2.0cm]{geometry}

% Referências e citações
\usepackage[
    pdftex,
    breaklinks,
    plainpages=false,
    pdfpagelabels,
    pagebackref,
    colorlinks=true,
    citecolor=DarkGreen,
    linkcolor=DarkBlue,
    urlcolor=DarkRed,
    filecolor=green,
    bookmarksopen=true
]{hyperref} 
\usepackage[all]{hypcap} % Soluciona o problema com o hyperref e capitulos
\usepackage[round,sort,nonamebreak]{natbib} % Citação bibliográfica plainnat-ime
%\bibpunct{(}{)}{;}{a}{\hspace{-0.7ex},}{,}  % Estilo de citação
\bibpunct{(}{)}{;}{a}{,}{,}

% Info
\title{MAC0214 - Proposta}
\author{
  \begin{tabular}{rl}
    Aluno:      & Victor Hugo Miranda Pinto \\
    Supervisor: & Renato Cordeiro Ferreira
  \end{tabular}
}
\date{2º Semestre de 2018}

\graphicspath{ {./img/} }
 
%%%%%%%%%%%%%%%%%%%%%%%%%%%%%%%%%%%%%%%%%%%%%%%%%%%%%%%%%%%%%%%%%%%%%%%%
\begin{document}

\maketitle
%%%%%%%%%%%%%%%%%%%%%%%%%%%%%%%%%%%%%%%%%%%%%%%%%%%%%%%%%%%%%%%%%%%%%%%%

\section{Introdução}
%%%%%%%%%%%%%%%%%%%%%%%%%%%%%%%%%%%%%%%%%%%%%%%%%%%%%%%%%%%%%%%%%%%%%%%%

  Introduzir as atividades que serão realizadas.


\section{Objetivos}
%%%%%%%%%%%%%%%%%%%%%%%%%%%%%%%%%%%%%%%%%%%%%%%%%%%%%%%%%%%%%%%%%%%%%%%%

    Ressaltar os objetivos dessas atividades a serem desenvolvidas.

    \begin{itemize}
      \item \textbf{Estabelecer um critério de seleção para o  HackathonUSP}\\
        Com essa pesquisa, espero estabelecer um critério de seleção que minimize o grau de desistência ao passo que maximize a diversidade (gênero, ano de ingresso, curso, unidade) entre os participantes do evento.

      \item \textbf{Segundo Objetivo}\\
        Após o desenvolvimento do algoritmo de seleção, desenvolver um serviço que faça a sugestão automática de participantes para o HackathonUSP pautada nos critérios citados acima.
    \end{itemize}
    
\section{Planejamento de atividades}
  
  Apresentar os passos de progresso para cada atividade e estimativas do número de horas.

  \begin{itemize}
    \item \textbf{Preparo dos dados a serem analisados} \\
        Nessa primeira fase da pesquisa, com previsão de durar por volta de um mês e meio, será feita a coleta e agregação do conjunto de dados que será utilizado para a pesquisa, além da normalização desses dados e a geração de visualizações para primeiros insights sobre os dados que tenho a disposição. Esses dados serão obtidos das inscrições nas edições passadas do HackathonUSP.
    
    \item \textbf{Desenvolvimento do algoritmo} \\
        Na segunda fase da pesquisa, seriam definidos os critérios preliminar de decisão para o algoritmo de seleção e a comparação de diferentes modelos segundo critérios de acurácia. Assim espero obter o algoritmo que será utilizado no primeiro teste da pesquisa em um conjunto diferente do conjunto de teste.
          
    \item \textbf{Desenvolvimento do serviço} \\
        Após decidir o algoritmo a ser utilizado sobre os futuros dados de teste, será iniciado o desenvolvimento de um módulo integrante da plataforma Hacknizer, desenvolvida pelo USPCodeLab, utilizando Flask para criar uma GraphQL API que receberá uma lista de inscritos para uma edição de um determinado Hackathon e, aplicando o algoritmo desenvolvido, poderá gerar uma lista sugerindo os participantes para tal Hackathon. 
    
    \item \textbf{Feedback} \\
        O segundo HackathonUSP de 2018 irá ocorrer no dia 02 de novembro. Planejo utilizar o serviço desenvolvido para obter uma sugestão dos participantes a serem chamados para o evento. Com isso, poderia obter feedback de toda a pesquisa realizada e do algoritmo criado, na perspectiva de organizador do evento e também dos participantes. Com esses resultados, será elaborado o resultado final da pesquisa para esse semestre. Ressaltando que o serviço gerado e o algoritmo desenvolvido continuará a ser aperfeiçoado.

  \end{itemize}

\section{Método de acompanhamento}

    Durante o semestre irei manter atualizada semanalmente a página <link> com atualizações sobre as atividades sendo desenvolvidas junto com o USPCodeLab. Nossos projetos e minhas contribuições nas implementações dessas atividades vão estar no repositório em <link>.
    Meu supervisor, Renato Cordeiro Ferreira é aluno de mestrado e pode ser contatado no e-mail: renatocf@ime.usp.br.

\end{document}