\documentclass[10pt,twoside,a4paper]{article}
% ^- openany - open new pages in odd/even page

%packages
\usepackage[T1]{fontenc}
\usepackage[utf8]{inputenc}      % Encoding
\usepackage[portuguese]{babel}   % Correção
%\usepackage{caption}             % Legendas
\usepackage{enumerate}

% Matemática
\usepackage{amsmath}             % Matemática
\usepackage{amsthm, amssymb}     % Matemática
\newtheorem*{def*}{Definição}
\newtheorem*{invariant}{Invariante}

% Gráficos
\usepackage[usenames,dvipsnames]{color}  % Cores
\usepackage[pdftex]{graphicx}   % usamos arquivos pdf/png como figura
\usepackage[usenames,svgnames,dvipsnames]{xcolor}

% Desenhos
\usepackage{tikz}
\usepgfmodule{decorations}
\usetikzlibrary{patterns}
\usetikzlibrary{decorations.shapes}
\usetikzlibrary{shapes.geometric}
\usetikzlibrary{decorations.text}
\usetikzlibrary{positioning} % Adjust grid size

% Código-fonte
\usepackage[noend]{algpseudocode}
\usepackage{algorithm}

% Configurações da página
\usepackage{fancyhdr}           % header & footer
\usepackage{float}
\usepackage{setspace}           % espaçamento flexível
\usepackage{indentfirst}        % Identa primeiro parágrafo
\usepackage{makeidx}
\usepackage[nottoc]{tocbibind}  % acrescentamos a  bibliografia/indice/
                                % conteudo no Table of Contents
                                
% Fontes
%\usepackage{helvet}
\renewcommand{\familydefault}{\sfdefault}
\usepackage{type1cm}            % fontes realmente escaláveis
\usepackage{titletoc}
\usepackage{pdflscape}          % Páginas em paisagem
\usepackage{pdfpages}

% Fontes e margens
\usepackage[fixlanguage]{babelbib}
\usepackage[font=small,format=plain,labelfont=bf,up,textfont=it,up]{caption}
\usepackage[a4paper,top=3.0cm,bottom=3.0cm,left=2.0cm,right=2.0cm]{geometry}

% Referências e citações
\usepackage[
    pdftex,
    breaklinks,
    plainpages=false,
    pdfpagelabels,
    pagebackref,
    colorlinks=true,
    citecolor=DarkGreen,
    linkcolor=DarkBlue,
    urlcolor=DarkRed,
    filecolor=green,
    bookmarksopen=true
]{hyperref} 
\usepackage[all]{hypcap} % Soluciona o problema com o hyperref e capitulos
\usepackage[round,sort,nonamebreak]{natbib} % Citação bibliográfica plainnat-ime
%\bibpunct{(}{)}{;}{a}{\hspace{-0.7ex},}{,}  % Estilo de citação
\bibpunct{(}{)}{;}{a}{,}{,}

% Info
\title{MAC0214 - Proposta}
\author{
  \begin{tabular}{rl}
    Aluno:      & Victor Hugo Miranda Pinto \\
    Supervisor: & Renato Cordeiro Ferreira
  \end{tabular}
}
\date{2º Semestre de 2018}

\graphicspath{ {./img/} }
 
%%%%%%%%%%%%%%%%%%%%%%%%%%%%%%%%%%%%%%%%%%%%%%%%%%%%%%%%%%%%%%%%%%%%%%%%
\begin{document}

\maketitle
%%%%%%%%%%%%%%%%%%%%%%%%%%%%%%%%%%%%%%%%%%%%%%%%%%%%%%%%%%%%%%%%%%%%%%%%

\section{Introdução}
%%%%%%%%%%%%%%%%%%%%%%%%%%%%%%%%%%%%%%%%%%%%%%%%%%%%%%%%%%%%%%%%%%%%%%%%

    Nesta proposta, apresento uma descrição de algumas atividades que irei desempenhar como membro do grupo de extensão USPCodeLab e que gostaria que fossem contabilizadas para a disciplina MAC0214.
    
    Conheci o grupo no começo de 2018 por meio de um dos seus fundadores, Leonardo Lana, com um convite para participar da \textit{USPCodeLab Summer School 2018, escola de verão do grupo}, na última semana do mês de Janeiro. Foi uma excelente experiência, onde aprendi muito e gostei das pessoas com quem trabalhei. Após a escola, fui convidado para a produção do grupo. Acabei sendo apontado como vice-presidente do USPCodeLab no mês de Março, quando lançamos o “USPCodeLab 2.0”, com novos formatos de atividades. 
    
    Desde então, tenho participado ativamente da parte administrativa do grupo. Organizamos diversos eventos para alunos de diversas unidades da USP, sempre com o objetivo de cumprir com a nossa missão: \textit{incentivar a inovação tecnológica na USP}. Dentre eles, incluem-se: o HackathonUSP 2018.1, nosso principal evento; e o Hackfools, um hackday (hackathon curto) com temática de primeiro de Abril. Estive presente em todos os eventos como organizador.

    Além dos nossos eventos, também mantivemos durante o semestre nossa iniciativa educational, o dev.journey(), que foi composta de três partes: o dev.start(), nosso ciclo de oficinas introdutórias sobre desenvolvimento web, incluindo palestras de HTML, CSS e JavaScript; o dev.learn(), grupo de estudos iniciante, para quem participou de um dos do dev.start; e o dev.boost(), grupo de estudos avançado, para quem já tive algum contato com desenvolvimento de sistemas web ou veio do dev.learn. Participei como monitor no dev.start e dev.learn além de ter atuado como desenvolvedor no dev.boost.



\section{Objetivos}
%%%%%%%%%%%%%%%%%%%%%%%%%%%%%%%%%%%%%%%%%%%%%%%%%%%%%%%%%%%%%%%%%%%%%%%%

    Além de manter minha participação como vice-presidente do grupo e na organização dos nossos eventos, gostaria de destacar três projetos nos quais irei me dedicar ao longo deste semestre, visando aumentar a capacidade do grupo de difundir conhecimentos e facilitar a criação de novos projetos:

    \begin{itemize}
      \item \textbf{MOOC de introdução ao desenvolvimento Web}\\
        MOOCs, do inglês \textit{Massive Open Online Courses}, são cursos disponibilizados na Web que buscam alcançar um grande número de alunos devido ao seu formato e facilidade de acesso. Neste projeto, que será desenvolvido com auxílio de outros membros do USPCodeLab, pretendemos criar um curso que introduza tecnologias e ferramentas essenciais para interessados em aprender sobre desenvolvimento Web.
        
        As ferramentas que serão cobertas durante o curso serão: arquitetura da Internet e da Web, HTML, CSS, JavaScript, Git e CDNs (\textit{Content Delivery Networks}, em particular o site Netlify). Com isso, o curso também será útil para qualquer interessado também em competições como Hackathons.
        
        Ao final do desenvolvimento do curso, vamos publicá-lo na página da Universidade de São Paulo na plataforma Coursera (https://www.coursera.org/usp) , que já possui uma coleção de cursos disponíveis. Isso será possível graças ao apoio do coordenador administrativo da Pró-Reitoria de Pesquisa da USP, Paulo Almeida, responsável pelos cursos da USP no Coursera. O USPCodeLab será responsável por criar uma trilha sobre desenvolvimento web e mobile. Assim, disponibilizaremos conteúdo de alta qualidade acessível para alunos e para a comunidade externa, suprindo uma demanda por cursos sobre essas tecnologias em Português.

        
        As ferramentas que serão cobertas durante o curso são: HTML, CSS, JavaScript, Git e Netlify, além de uma introdução sobre o protocolo HTTP e como a Web funciona. Com isso, o curso é também útil para qualquer interessado em desenvolvimento Web, mesmo que não visando participação em Hackathons.
        
        Ao final do desenvolvimento do curso, vamos publicá-lo na plataforma Coursera sob o nome da Universidade de São Paulo, que já possui uma coleção de cursos disponíveis. Isso será possível graças ao coordenador administrativo da Pró-Reitoria de Pesquisa da USP, Paulo Almeida, responsável pelos cursos da USP no Coursera, com quem combinamos que o USPCodeLab seria responsável por criar uma \textit{trilha} de cursos sobre desenvolvimento web e mobile para a plataforma. Assim, estamos disponibilizando conteúdo de alta qualidade facilmente acessível para alunos e não-alunos da universidade, além de suprir uma demanda por cursos de tecnologia em Português.

      \item \textbf{MOOC sobre APIs REST com Node.js}\\
        Ainda como parte da trilha de desenvolvimento web e mobile, será desenvolvido um curso com foco no back-end de sistemas web, em específico o desenvolvimento de APIs de acordo com o padrão REST (\textit{Representational State Transfer}).
        
        Este curso irá introduzir Node.js aos alunos, mostrando como JavaScript pode ser utilizado fora do browser para criar servidores. Além disso, tratará sobre técnicas para construção de APIs REST. Ao longo do curso, o aluno será guiado na construção de uma API que segue esse padrão escrita em JavaScript.
        
      \item \textbf{CodeLab-CLI}\\
        O CodeLab-CLI é um projeto de um utilitário de linha de comando, baseado em \textit{Node.js} para facilitar a criação de novos projetos de software utilizando \textit{templates} customizáveis para projetos que utilizam as tecnologias que mais usamos no USPCodeLab.
        
        Esse utilitário permitiria criação de novos projetos mais rapidamente, removendo a necessidade de configuração inicial pelo desenvolvedor, podendo ser utilizada para criar APIs (REST ou GraphQL) ou clientes (SPAs ou PWAs) com uma série de pré-configurações definidas como padrão dentro do grupo, porém sem perder a customização, pois o desenvolvedor iria escolher uma série de opções disponíveis para tais configurações.
        

    \end{itemize}
    
\section{Planejamento de atividades}
  
  Seguem as estimativas do tempo de trabalho em cada etapa de cada uma das atividades.

  \begin{itemize}
    \item \textbf{MOOC de introdução ao desenvolvimento Web - 50 horas} \\
        O curso será baseado na parte teórica do que se é abordado no grupo de estudos iniciante dev.learn(). Com isso, a primeira parte do desenvolvimento do curso será adaptar o material que já existente (disponível em https://uclab.xyz/slides) para o formato que planejamos para o MOOC. Após a adaptação, precisamos escrever o roteiro do curso, ou seja, definir a ordem e a maneira com que cada tópico será apresentado ao longo do curso.

        Após essa etapa preliminar, o próximo passo será a gravação das vídeo-aulas, que será seguido pela edição e revisão dos vídeos. Estamos considerando um curso com aproximadamente 10 horas de duração Baseado nessa estimativa, calculamos as seguintes durações  para cada etapa desse projeto:

        
        \begin{center}
            \begin{tabular}{ |l|c| } 
                \hline
                     Adaptação do material existente & 5 horas \\
                     \hline
                     Idealização e escrita do roteiro & 10 horas \\
                     \hline
                     Gravação do material de vídeo & 20 horas \\
                     \hline
                     Edição e revisão dos vídeos & 15 horas \\
                \hline
            \end{tabular}
        \end{center}
    
    \item \textbf{MOOC sobre APIs REST com Node.js - 45 horas} \\
        % - 10h preparação de material
        % - 10h de roteiro
        % - 15h de gravação
        % - 15h edição (com revisão)
        Para este curso, não temos material previamente disponível. Logo, a primeira etapa de desenvolvimento incluirá desenvolver um repositório com o projeto a ser feito durante o curso (API REST em JavaScript). Após o desenvolvimento e teste, as etapas são similares às do outro MOOC, sendo que será preciso montar o roteiro das aulas, gravá-las, editá-las e revisá-las. Para esse curso, também estamos considerando 10 horas de vídeo. Mais uma vez, baseado nessa estimativa,segue a duração estimada de cada etapa:
        
        \begin{center}
            \begin{tabular}{ |l|c| } 
                \hline
                     Criação do material e projeto do curso & 10 horas \\
                     \hline
                     Idealização e escrita do roteiro & 10 horas \\
                     \hline
                     Gravação do material de vídeo &  15 horas \\
                     \hline
                     Edição e revisão dos vídeos & 10 horas \\
                \hline
            \end{tabular}
        \end{center}
        
    \item \textbf{CodeLab-CLI - 10 horas} \\
        
        
  \end{itemize}
  
\section{Cronograma}

    \begin{center}
            \begin{tabular}{ |l|l|c| } 
                \hline
                    MOOC de introdução ao desenvolvimento Web
                     & Adaptação do material existente & 5 horas \\
                     \hline
                     & Idealização e escrita do roteiro & 10 horas \\
                     \hline
                     & Gravação do material de vídeo & 20 horas \\
                     \hline
                     & Edição e revisão dos vídeos & 15 horas \\
                \hline
                    MOOC sobre APIs REST com Node.js
                     & Criação do material e projeto do curso & 10 horas \\
                     \hline
                     & Idealização e escrita do roteiro & 10 horas \\
                     \hline
                     & Gravação do material de vídeo &  15 horas \\
                     \hline
                     & Edição e revisão dos vídeos &  10 horas \\
                \hline
                    CodeLab-CLI
                     &  &  horas \\
                     \hline
                     &  &  horas \\
                     \hline
                     &  &  horas \\
                     \hline
                     &  &  horas \\
                
                \hline
                    Total
                    & & 105 horas \\
                \hline
            \end{tabular}
    \end{center}

\section{Método de acompanhamento}

    Durante o semestre irei manter atualizada semanalmente a página \url{https://victorhmp.github.io/mac0214-blog/} com atualizações sobre as atividades sendo desenvolvidas junto com o USPCodeLab. Nossos projetos e minhas contribuições nas implementações dessas atividades vão estar no repositório em \url{https://gitlab.com/uspcodelab/}.
    Meu supervisor será o aluno de mestrado e coordenador do USPCodeLab, Renato Cordeiro Ferreira.. Ele pode ser contatado pelo e-mail: renatocf@ime.usp.br.

\end{document}