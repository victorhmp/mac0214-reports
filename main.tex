\documentclass[10pt,twoside,a4paper]{article}
% ^- openany - open new pages in odd/even page

%packages
\usepackage[T1]{fontenc}
\usepackage[utf8]{inputenc}      % Encoding
\usepackage[portuguese]{babel}   % Correção
%\usepackage{caption}             % Legendas
\usepackage{enumerate}

% Matemática
\usepackage{amsmath}             % Matemática
\usepackage{amsthm, amssymb}     % Matemática
\newtheorem*{def*}{Definição}
\newtheorem*{invariant}{Invariante}

% Gráficos
\usepackage[usenames,dvipsnames]{color}  % Cores
\usepackage[pdftex]{graphicx}   % usamos arquivos pdf/png como figura
\usepackage[usenames,svgnames,dvipsnames]{xcolor}

% Desenhos
\usepackage{tikz}
\usepgfmodule{decorations}
\usetikzlibrary{patterns}
\usetikzlibrary{decorations.shapes}
\usetikzlibrary{shapes.geometric}
\usetikzlibrary{decorations.text}
\usetikzlibrary{positioning} % Adjust grid size

% Código-fonte
\usepackage[noend]{algpseudocode}
\usepackage{algorithm}

% Configurações da página
\usepackage{fancyhdr}           % header & footer
\usepackage{float}
\usepackage{setspace}           % espaçamento flexível
\usepackage{indentfirst}        % Identa primeiro parágrafo
\usepackage{makeidx}
\usepackage[nottoc]{tocbibind}  % acrescentamos a  bibliografia/indice/
                                % conteudo no Table of Contents
                                
% Fontes
%\usepackage{helvet}
\renewcommand{\familydefault}{\sfdefault}
\usepackage{type1cm}            % fontes realmente escaláveis
\usepackage{titletoc}
\usepackage{pdflscape}          % Páginas em paisagem
\usepackage{pdfpages}

% Fontes e margens
\usepackage[fixlanguage]{babelbib}
\usepackage[font=small,format=plain,labelfont=bf,up,textfont=it,up]{caption}
\usepackage[a4paper,top=3.0cm,bottom=3.0cm,left=2.0cm,right=2.0cm]{geometry}

% Referências e citações
\usepackage[
    pdftex,
    breaklinks,
    plainpages=false,
    pdfpagelabels,
    pagebackref,
    colorlinks=true,
    citecolor=DarkGreen,
    linkcolor=DarkBlue,
    urlcolor=DarkRed,
    filecolor=green,
    bookmarksopen=true
]{hyperref} 
\usepackage[all]{hypcap} % Soluciona o problema com o hyperref e capitulos
\usepackage[round,sort,nonamebreak]{natbib} % Citação bibliográfica plainnat-ime
%\bibpunct{(}{)}{;}{a}{\hspace{-0.7ex},}{,}  % Estilo de citação
\bibpunct{(}{)}{;}{a}{,}{,}

% Info
\title{MAC0214 - Proposta}
\author{
  \begin{tabular}{rl}
    Aluno:      & Victor Hugo Miranda Pinto \\
    Supervisor: & Renato Cordeiro Ferreira
  \end{tabular}
}
\date{2º Semestre de 2018}

\graphicspath{ {./img/} }
 
%%%%%%%%%%%%%%%%%%%%%%%%%%%%%%%%%%%%%%%%%%%%%%%%%%%%%%%%%%%%%%%%%%%%%%%%
\begin{document}

\maketitle
%%%%%%%%%%%%%%%%%%%%%%%%%%%%%%%%%%%%%%%%%%%%%%%%%%%%%%%%%%%%%%%%%%%%%%%%

\section{Introdução}
%%%%%%%%%%%%%%%%%%%%%%%%%%%%%%%%%%%%%%%%%%%%%%%%%%%%%%%%%%%%%%%%%%%%%%%%

  Nessa proposta apresento uma descrição de algumas das atividades que irei desempenhar como membro do grupo de extensão USPCodeLab e gostaria que fossem contabilizadas para a disciplina MAC0214.
  
  Conheci o grupo no começo de 2018 através de um dos seus fundadores, Leonardo Lana, com um convite para participar da USPCodeLab Summer School 2018 na última semana do mês de janeiro. Foi uma excelente experiência, onde aprendi muito e gostei muito das pessoas com quem trabalhei. Após a escola fui convidado para a organização do grupo e aceitei, sendo apontado como vice-presidente no mês de março quando lançamos o que chamamos de USPCodeLab 2.0. Com isso, tenho participado ativamente da parte administrativa do grupo.
  
  Desde então, organizamos diversos eventos para alunos da USP, sempre com o objetivo de incentivar a inovação e difundir o conhecimento que obtemos referente a desenvolvimento de software para a Web. Alguns deles foram: HackathonUSP 2018.1, nosso principal evento; o dev.start(), nosso ciclo de palestras de HTML, CSS e JavaScript para iniciantes que tem como objetivo introduzir os alunos ao mundo do desenvolvimento web; Hackfools, um hackday com temática de primeiro de abril. Estive presente em todos os eventos como organizador.
  
  Além dos nossos eventos, também mantivemos durante o semestre inteiro nossos dois grupos de estudo, o dev.learn() para alunos iniciantes (em geral, quem participou do dev.start()) e o dev.boost() para alunos avançados que já tiveram contato com desenvolvimento de sistemas de software web. Participei do dev.learn() como monitor e no dev.boost() como desenvolvedor.


\section{Objetivos}
%%%%%%%%%%%%%%%%%%%%%%%%%%%%%%%%%%%%%%%%%%%%%%%%%%%%%%%%%%%%%%%%%%%%%%%%

    Além de manter minha participação como vice-presidente do grupo e na organização dos nossos eventos, gostaria de destacar dois projetos nos quais irei me dedicar principalmente ao longo desse semestre visando aumentar a capacidade do grupo de difundir conhecimento:

    \begin{itemize}
      \item \textbf{MOOC de preparação para Hackathons (HackMOOC)}\\
        MOOCs são \textit{Massive Open Online Courses}, cursos disponibilizados na Web que buscam alcançar um grande número de alunos devido ao seu formato e disponibilidade. Nesse projeto, que será desenvolvido com outros membros do USPCodeLab, pretendemos criar um curso que introduza tecnologias e ferramentas úteis (e talvez essenciais) para interessados em participar de Hackathons.
        
        As ferramentas que serão cobertas durante o curso são: HTML, CSS, JavaScript, Git e Netlify, além de uma introdução sobre o protocolo HTTP e como a Web funciona. Com isso, o curso é também útil para qualquer interessado em desenvolvimento Web, mesmo que não visando participação em Hackathons.
        
        Ao final do desenvolvimento do curso, vamos publicá-lo na plataforma Coursera sob o nome da Universidade de São Paulo, que já possui uma coleção de cursos disponíveis. Isso será possível graças ao coordenador administrativo da Pró-Reitoria de Pesquisa da USP, Paulo Almeida, responsável pelos cursos da USP no Coursera, com quem combinamos que o USPCodeLab seria responsável por criar uma \textit{trilha} de cursos sobre desenvolvimento web e mobile para a plataforma. Assim, estamos disponibilizando conteúdo de alta qualidade facilmente acessível para alunos e não-alunos da universidade, além de suprir uma demanda por cursos de tecnologia em Português.

      \item \textbf{MOOC sobre APIs REST com Node.js}\\
        Seguindo o desenvolvimento da trilha de cursos sobre desenvolvimento web e mobile, será desenvolvido um curso com foco no \textit{back-end} de sistemas web, em específico o desenvolvimento de APIs de acordo com o padrão REST.
        
        Esse curso pretende introduzir Node.js aos alunos, mostrando como JavaScript pode ser usado fora do browser em servidores, e o padrão REST de APIs, de tal forma que ao longo do curso o aluno seja guiado na criação de uma API REST escrita em JavaScript, usando Koa como framework.
    \end{itemize}
    
\section{Planejamento de atividades}
  
  Seguem as estimativas do tempo de trabalho em cada etapa de cada uma das atividades.

  \begin{itemize}
    \item \textbf{MOOC de preparação para Hackathons (HackMOOC) - 50 horas} \\
        O curso será baseado na parte teórica do que se é abordado no grupo de estudos iniciante dev.learn(). Com isso, a primeira parte do desenvolvimento do curso será adaptar o material que já temos para o formato que planejamos para o MOOC. Após a adaptação, precisamos escrever o roteiro do curso, ou seja, definir a ordem e a maneira com que cada tópico será apresentado ao longo do curso. Isso conclui a etapa preliminar de criação do curso, sendo o próximo passo a gravação das video-aulas. Após concluídas as gravações, a etapa final de desenvolvimento do curso será a de edição e revisão dos vídeos. Estamos considerando um curso com 10 horas de duração de vídeo, com isso em mente, segue a duração de cada etapa desse desenvolvimento:
        
        \begin{center}
            \begin{tabular}{ |l|c| } 
                \hline
                     Adaptação do material existente & 5 horas \\
                     \hline
                     Idealização e escrita do roteiro & 10 horas \\
                     \hline
                     Gravação do material de vídeo & 20 horas \\
                     \hline
                     Edição e revisão dos vídeos & 15 horas \\
                \hline
            \end{tabular}
        \end{center}
        
        Assim, totalizando 50 horas de trabalho.
    
    \item \textbf{MOOC sobre APIs REST com Node.js - 50 horas} \\
        % - 10h preparação de material
        % - 10h de roteiro
        % - 15h de gravação
        % - 15h edição (com revisão)
        Para esse curso, não temos material previamente feito, logo a primeira etapa de desenvolvimento do curso será desenvolver um repositório com o projeto a ser feito durante o curso, já que o curso terá como objetivo a criação de uma API REST funcional ao ser concluído. Após o desenvolvimento e teste dessa API que servirá como "material" do curso, é preciso montar o roteiro das aulas do curso. Com o roteiro e o material pronto, a próxima etapa é a de gravação das video-aulas. Após concluídas as gravações, a etapa final, como no MOOC anterior, será a de edição e revisão das aulas. Estou planejando 10 horas de vídeo para esse curso também, com isso em mente, segue a duração de cada etapa:
        
        \begin{center}
            \begin{tabular}{ |l|c| } 
                \hline
                     Criação do material e projeto do curso & 10 horas \\
                     \hline
                     Idealização e escrita do roteiro & 10 horas \\
                     \hline
                     Gravação do material de vídeo & 15 horas \\
                     \hline
                     Edição e revisão dos vídeos & 15 horas \\
                \hline
            \end{tabular}
        \end{center}

        Assim, totalizando mais 50 horas de trabalho.
  \end{itemize}

\section{Método de acompanhamento}

    Durante o semestre irei manter atualizada semanalmente a página \url{https://victorhmp.github.io/mac0214-blog/} com atualizações sobre as atividades sendo desenvolvidas junto com o USPCodeLab. Nossos projetos e minhas contribuições nas implementações dessas atividades vão estar no repositório em \url{https://gitlab.com/uspcodelab/}.
    Meu supervisor, Renato Cordeiro Ferreira, é aluno de mestrado e pode ser contatado no e-mail: renatocf@ime.usp.br.

\end{document}